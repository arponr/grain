
\input{preamble}

\title{Herchel-Smith Fellowship Midyear Report}
\author{Arpon Raksit}
\date{\today}

\numberwithin{equation}{subsection}
% \cspreto{section}{\setcounter{equation}{0}}

\begin{document}
\maketitle

%%%%%%%%%%%%%%%%%%%%%%%%%%%%%%%%%%%%%%%%%%%%%%%%%%%%%%%%%%%%%%%%%%%%%%

\section*{Academic growth}

I spent the week after Michaelmas term traveling around Andaluc\'{i}a with two other Harvardians (also on fellowship at Cambridge this year)---a day in Seville, two days in C\'{o}rdoba, two days in Granada, and one more day in Seville, the segments linked by quiet bus rides and capped by Ryanair flights costing 40 pounds round-trip, a price nothing short of miraculous to any American\footnote{Similarly miraculous to any American is the cost of operating a cell phone around here.}. During the trip I read the novel \emph{Leaving the Atocha Station}, by Ben Lerner; the book follows a young American poet who is in Madrid\footnote{But he briefly visits Granada, which in fact occurred wonderfully mimetically in my reading, during our bus ride to Granada!} on a fellowship, ostensibly funding a research project on the Spanish Civil War, but in reality funding prodigious consumption of hash and weed and perpetual contemplation of his---and his poetry's, and all poetry's---fraudulence and failure.

I'm happy to report that this is not how my fellowship in Cambridge is proceeding. There are, however, aspects of his experience that I can identify with. Indeed, I'd like to frame the growth I've experienced as a student this year by the following quotation:
\begin{quote}
  I didn't need to establish a life in Madrid beyond the simplest routines; I didn't have to worry about building a community, whatever that meant, I had the endless day, months and months of endless days, and yet my return date bounded this sense of boundlessness, kept it from becoming threatening. I would begin to feel a rush of what I considered love ... most intensely love for \emph{that other thing}, the sound-absorbent screen, life's white machine, shadows massing in the middle distance, although that's not even close, the texture of et cetera itself.\footnote{Ben Lerner, \emph{Leaving the Atocha Station} (Granta Books 2013), pp. 15--16.}
\end{quote}
This excerpt characterizes the mood of the novel as a whole: drifting, with ``shadows'' and ``textures'' of thoughts between moments more than clear-cut events or plot.

I think this characterizes the mood of my year so far in Cambridge as well, in the sense that I've had lots and lots of structureless time to myself. I don't have problem sets strictly due every week and therefore have much greater control over my time and  work this year than I did as an undergraduate. This is both a tremendous freedom and responsibility: it's given me a great opportunity to learn the things I really want to learn, but it's also given me a first glimpse into the discipline and maturity one needs as a graduate student. In particular, I've been coming to grips with the balance I personally need to maintain between productivity and relaxation, a balance trickier to strike with less structure and more freedom.

In other words, my academic life and growth has followed a sort of plotless trajectory, but as in \emph{Atocha}, this plotlessness does not imply stagnation.
\begin{quote}
  Not the little lyric miracles and luminous branching injuries, but the other thing, whatever it was, was life, and was falsified by any way of talking or writing or thinking that emphasized sharply localized occurrences in time.\footnote{\emph{Atocha}, p. 64.}
\end{quote}
The growing has been an accumulation of \emph{that other thing}, a gradual result of a lot of slow, deliberate thought---and one trip to southern Spain.


%%%%%%%%%%%%%%%%%%%%%%%%%%%%%%%%%%%%%%%%%%%%%%%%%%%%%%%%%%%%%%%%%%%%%%

\section*{Personal growth}

In my junior year of high school I decided to start using British English spellings. Who really knows why I did it, but I think it made me feel interesting and refined; I think I felt that ``American'' was an antonym of ``refined'', and ``British'' at least much closer to a synonym. I only switched back to fully American English in my senior year at Harvard, when a couple of my classes motivated me to consider what it might mean to be an American, and sort of awakened me to the fact that I very much am an American (whatever that might mean).

Arriving in and spending time in England has actually only sharpened this sense of American identity. Indeed, probably the most difficult adjustment my brain had to make in its first few days here was not to the British accent, or to the British temperament, but to the total upheaval of the brands of everyday products and stores in my life. So many new questions arose: First of all, which stores do I go to? Where's CVS and Walgreen's and Target? And when I get inside the strange alternate-dimension-like British equivalents, what am I supposed to buy? Is Lynx soap the same as Axe soap? They sure look the same so why doesn't it say Axe? Where is Tide detergent? What is happening? Missing my American brands, and, moreover, feeling a visceral identification of culture and consumer goods, I had never felt more American.

Perhaps correlated with my heightened feeling of Americanness, this is also the first year I've paid any attention to (American) politics. More generally, I've started reading and thinking about what's going on in the world instead of just in my life and my head, which feels like personal growth. I think this has more to do with some personal experiences and getting older than with being in England, but nevertheless it's amusing that my ties to American culture and citizenship have strengthened during my first long-term departure from the country.

I think I've grown in other ways too, but it's a little hard to describe. It probably has to do with leaving the extreme comfort of Harvard and its bubble, and feeling a tad more ``out in the world'' now. One way I can summarize the impression that these kinds of personal growth have culminated in is in the following (somewhat sophomoric, but I guess that's the point) anecdote: After my week in Spain, I flew home to New York for winter break. My loving parents picked me up at JFK, and I climbed into the back seat of my mom's car like I've done ten million times before. The difference this time was that, as we pulled onto the highway, though it certainly felt like home, it also felt strangely incongruous. As a child in the back seat staring at the backs of his parents' heads I felt misplaced. I told my parents that, and we decided that maybe I had actually grown up a little.

%%%%%%%%%%%%%%%%%%%%%%%%%%%%%%%%%%%%%%%%%%%%%%%%%%%%%%%%%%%%%%%%%%%%%%

\section*{Mathematical work}

With nebulous reflections about life and my growth as a human being out of the way, let me say something concrete about the math I've had the chance to learn this year.

One of my mathematical goals for this year was to step back, firm up, and broaden my understanding in areas of math that I didn't focus on at Harvard. It's panned out slightly differently than I had first planned, but I think I've still succeeded in this goal. In my first term I took classes in differential geometry, functional analysis, category theory, and probabilistic number theory. The first two are foundational subjects that I should have taken at Harvard but didn't get the chance to, so certainly factored nicely into the above goal. The last was actually my most interesting course in Michaelmas, and far different than anything I had learned about at Harvard. The course essentially investigated some analogies that allow one to (heuristically) study prime numbers as if they are randomly distributed among all whole numbers. The techniques were fairly elementary and there was lots of nice intuition, which all in all made for a very fun course.

The main plan that didn't pan out was my goal to learn some physics while I was here. I sat in the first several classes of two physics courses: one turned out to be really poorly taught (at least relative to my tastes), and the other began with some pure math that I already knew fairly well, so when I decided to quit attending the first I did the same with the second. However, I still did some independent reading during the term on some physics-related math which I had wanted to learn, about some objects called ``topological quantum field theories''. I ended up giving a thirty-minute talk on a bit of this reading to some other math students at the end of the term.

This term I'm just taking two courses, one on complex manifolds and one on elliptic curves, which allows me to spend more time on my ``essay'' (akin to an undergraduate thesis) and some other independent reading. The courses are on topics which are totally ubiquitous in math, at the intersection of many different areas. In fact, my essay topic, Hodge theory, ties very nicely in with both courses. I'm having a lot of fun with the essay so let me briefly try to describe what Hodge theory is about.

A prevalent philosophy in mathematics is to try to understand interesting geometric objects by attaching more easily manipulable algebraic invariants. For now, let's imagine ``geometric objects'' as certain subsets of Euclidean space of any dimension, for example a ($1$-dimensional) circle lying inside the ($2$-dimensional) plane. I won't try to explain what ``algebraic invariant'' means, but will just say there's a very useful one, called \emph{cohomology}; whenever we have a subset of Euclidean space, we can try to compute its cohomology to try to understand what properties this subset has, e.g. to detect whether it contains any holes. Hodge theory falls within the scope of algebraic geometry, which studies a special class of geometric objects, called \emph{algebraic varieties}, namely the subsets of Euclidean spaces obtained by considering the zeros of polynomial equations. For instance, the circle is one of these special subsets, since it can be described as the set of points $(x,y)$ in the plane where $x^2 + y^2 = 1$, or in other words where the polynomial $x^2 + y^2 - 1$ is zero. It turns out that the cohomology of algebraic varieties has special structure that the cohomology of arbitrary geometric objects does not have, and this extra structure gives us more useful information about varieties. The description and study of this structure is what Hodge theory is all about. It's a really fascinating story in math, and learning and writing about it is a great chance to gain some intuition and familiarity with important ideas in the field of algebraic geometry.

%%%%%%%%%%%%%%%%%%%%%%%%%%%%%%%%%%%%%%%%%%%%%%%%%%%%%%%%%%%%%%%%%%%%%%

% \bibliographystyle{amsalpha}
% \bibliography{refs}

\end{document}
