\section{Hodge--de Rham degeneration}
\label{degen}

The  goal of this section is to present Deligne-Illusie's algebraic argument for the following theorem.

\begin{theorem}
  \label{degen--main}
  Suppose $K$ is a field of characteristic zero and $X$ is a smooth, proper $K$-scheme. Then the Hodge--de Rham spectral sequence 
  \[
    E_1^{i,j} = \rH^j(X;\Omega_{X/K}^i)
    \quad \Rightarrow \quad
    \rH_\dR^{i+j}(X/K)
  \]
  degenerates immediately at $E_1$.
\end{theorem}

\begin{nothing}
  \label{degen--strategy}
  Let's begin with a rough intuition for the Deligne-Illusie strategy. First, note that, since $X$ is proper, the Hodge--de Rham spectral sequence is a spectral sequence of \emph{finite-dimensional} $K$-vector spaces, so we always have the inequality
  \[
    \dim_K \rH_\dR^k(X/K) \le \sum_{i+j=k} \dim_K \rH^j(X;\Omega_{X/K}^i),
  \]
  with equality holding if and only if the spectral sequence degenerates at $E_1$. In particular, to prove degeneration it would suffice to give any isomorphism
  \[
    \rH_\dR^k(X/K) \iso \bigoplus_{i+j=k} \rH^j(X;\Omega_{X/K}^i).
  \]
  The left-hand side is by definition the $k$-th sheaf cohomology group of the de Rham complex $\Omega_{X/K}^\bullet$; if we view the sheaf $\Omega_{X/K}^i$ as a cochain-complex concentrated in degree $i$, then the right-hand side may be reformulated as the $k$-th sheaf cohomology group of the cochain-complex $\bigoplus_i \Omega_{X/K}^i[-i]$, which has the same components as the de Rham complex but trivial differential. It would thus suffice to give an equivalence
  \begin{equation}
    \label{degen--strategy--desire-lift}
    \Omega_{X/K}^\bullet \iso \bigoplus_i \Omega_{X/K}^i[-i]
  \end{equation}
  in the derived category of $X$. Of course, for this to exist, the cohomology sheaves of these two complexes would need to be isomorphic, i.e. we would need
  \begin{equation}
    \label{degen--strategy--desire}
    \rH^i(\Omega_{X/K}^\bullet) \iso \Omega_{X/K}^i.
  \end{equation}

  To investigate the plausibility of this statement, let's consider a very simple example: $X \ce \lA^1_K = \Spec K[t]$. As $X$ is affine, we may think of quasi-coherent sheaves on $X$ simply as their $K[t]$-modules of global sections. The sheaf $\Omega_{X/K}^1$ is free of rank $1$, generated by $dt$, so the de Rham complex is concentrated in degrees $0$ and $1$, given simply by the differentiation map
  \[
    K[t] \lblto{d} K[t]dt.
  \]
  The key observation is that this complex behaves very differently in positive characteristic than it does in characteristic zero.
  
  If $K$ has characteristic $0$, $d$ is surjective and the kernel of $d$ is given simply by the constants $K$, so we recover the usual cohomology of affine space:
  \[
    \rH^0(\Omega_{X/K}^\bullet) = K, \quad
    \rH^1(\Omega_{X/K}^\bullet) = 0,
  \]
  so \cref{degen--strategy--desire} certainly does not hold.

  If $K$ has characteristic $p > 0$, then the fact that $t^p$ has derivative $pt^{p-1} = 0$ implies that the kernel of $d$ consists of polynomials of the form $f(t^p)$, and the cokernel of $d$ consists of $1$-forms of the form $t^{p-1}f(t)dt$. That is we have,
  \[
    \rH^0(\Omega_{X/K}^\bullet) = K[t^p] \iso K[t], \quad \rH^0(\Omega_{X/K}^\bullet) = t^{p-1}K[t]dt \iso K[t]dt,
  \]
  which suggests that, modulo the intervention of the $p$-th-powers, something like \cref{degen--strategy--desire} may indeed hold!

  The conclusion is that the strategy outlined above may yield degeneration results in positive characteristic. This is indeed the case. In \cref{degen--cartier} we will formulate the precise version of the isomorphism \cref{degen--strategy--desire} that holds in characteristic $p> 0$, taking into account the $p$-th powers. In \cref{degen--lift} we will study when this isomorphism may be lifted to an equivalence \cref{degen--strategy--desire-lift} in the derived category, and show that when such a lift exists we obtain a degeneration result. Finally, in \cref{degen--final} we will indicate how these positive characteristic results in fact imply the desired characteristic zero result.
\end{nothing}

%%%%%%%%%%%%%%%%%%%%%%%%%%%%%%%%%%%%%%%%%%%%%%%%%%%%%%%%%%%%%%%%%%%%%%

\subsection{The Cartier isomorphism}
\label{degen--cartier}

Throughout this subsection and the next we fix a prime $p$. We first discuss $p$-th power maps.

\begin{definition}
  \label{degen--cartier--frobenius}
  \begin{enumerate}[leftmargin=*]
  \item We say a scheme $X$ is of \emph{characteristic $p$} if $p\sO_X = 0$, or equivalently if the unique map $X \to \Spec \lZ$ factors through $\Spec \lF_p$.
  \item If $X$ is a scheme of characteristic $p$, then there is an \emph{ absolute Frobenius map} $F_X \c X \to X$ which is the identity map on the underlying topological space of $X$ and is given by the $p$-th power map $a \mapsto a^p$ on the structure sheaf $\sO_X$.
  \item Let $f \c X \to S$ be a map of schemes of characteristic $p$. The diagram
    \begin{equation}
      \label{degen--cartier--frobenius--commute}
      \begin{tikzcd}
        X \ar[r, "F_X"] \ar[d, "f"] &
        X \ar[d, "f"] \\
        S \ar[r, "F_S"] &
        S
      \end{tikzcd}
    \end{equation}
    evidently commutes. Define the scheme $X'$ (also denoted $X'_S$) and maps $G_{X/S} \c X' \to X$ and $f' \c X' \to S$ by the pullback diagram
    \[
      \begin{tikzcd}
        X' \ar[r, "G_{X/S}"] \ar[d, "f'"] &
        X \ar[d, "f"] \\
        S \ar[r, "F_S"] &
        S,
      \end{tikzcd} 
    \]
    and define the \emph{relative Frobenius map} $F_{X/S} \c X \to X'$ to be the map determined by \cref{degen--cartier--frobenius--commute}.
  \end{enumerate}
\end{definition}

\begin{situation}
  For the remainder of this subsection and the next, let $f \c X \to S$ be a map of schemes of characteristic $p$.
\end{situation}

\begin{lemma}
  \label{degen--cartier--frobenius-homeo}
  At the level of topogical spaces, the maps $F_{X/S}$ and $G_{X/S}$ are inverse homeomorphisms. In addition, $F_{X/S}$ is universally injective (i.e. injective after any base change).
\end{lemma}

\begin{proof}
  By definition we have $G_{X/S}F_{X/S} = F_X$, and one easily checks that $F_{X/S}G_{X/S} = F_{X'}$. Since the absolute Frobenius map is by definition the identity on topological spaces, it follows that $F_{X/S}$ and $G_{X/S}$ are homeomorphisms. Since $G_{X/S}$ is an arbitrary base change of $F_S$, this proves that an absolute Frobenius map is a universal homeomorphism. Thus $G_{X/S}F_{X/S} = F_X$ is a universal homeomorphism, proving $F_{X/S}$ is universally injective.
\end{proof}

\begin{nothing}
  \label{degen--cartier--frobenius-id}
  By \cref{degen--cartier--frobenius-homeo}, we can think of $X'$ as having the same underlying topological space as $X$. Indeed, in what follows we will make this (perhaps abusive) identification, which allows us to omit the pushforward and inverse-image functors $(F_{X/S})_*,F_{X/S}^{-1},(G_{X/S})_*,G_{X/S}^{-1}$ from our notation.

  The structure sheaf $\sO_{X'}$ is then the sheaf on $X$ given by $f^{-1}\sO_S \otimes_{f^{-1}\sO_S}\sO_X $, where the left-hand $f^{-1}\sO_s$ is viewed as a module over the base $f^{-1}\sO_S$ via the Frobenius $F_S$. So sections of $\sO_{X'}$ are locally of the form $a \otimes t$ for $a$ a section of $f^{-1}\sO_S$ and $t$ a section of $\sO_X$. In these terms:
  \begin{itemize}
  \item $G_{X/S} \c X' \to X$ is given by the map $G_{X/S}^\sharp \c \sO_X \to \sO_{X'}$ sending $at \mapsto a^p \otimes t$;
  \item $F_{X/S} \c X \to X'$ is given be the map $F_{X/S}^\sharp \c \sO_{X'} \to \sO_X$ sending $a \otimes t \mapsto at^p$.
  \end{itemize}
  This perhaps is clearest in the following example.
\end{nothing}

\begin{example}
  \label{degen--cartier--frobenius-eg}
  Let $S \ce \Spec A$ for a ring $A$ of characteristic $p$ and $X \ce \lA^n_S = \Spec A[t_1,\ldots,t_n]$, and consider the canonical map $f \c X \to S$. The scheme $X'$ is also isomorphic to $\lA^n_S$ and:
  \begin{itemize}
  \item the map $G_{X/S} \c X' \to X$ corresponds to the map $A[t_1,\ldots,t_n] \to A[t_1,\ldots,t_n]$ sending $at_i \mapsto a^pt_i$ for $a \in A$;
  \item the relative Frobenius $F_{X/S} \c X \to X'$ corresponds to the map $A[t_1,\ldots,t_n] \to A[t_1,\ldots,t_n]$ sending $at_i \mapsto at_i^p$ for $a \in A$.
  \end{itemize}
\end{example}

\begin{lemma}
  \label{degen--cartier--frobenius-etale}
  Suppose $f$ is \'{e}tale. Then the relative Frobenius $F_{X/S}$ is an isomorphism.
\end{lemma}

\begin{proof}
  Since $f$ is \'{e}tale, so is its base-change $f' \c X' \to S$, and therefore so is $F_{X/S}$. Since $F_{X/S}$ is also universally injective \cref{degen--cartier--frobenius-homeo} it must be an open embedding. But \cref{degen--cartier--frobenius-homeo} also tells us it's a homeomorphism, so indeed it must be an isomorphism.
\end{proof}

\begin{lemma}
  \label{degen--cartier--frobenius-smooth}
  Suppose $f$ is smooth of relative dimension $n$. Then $\sO_X$ is locally free of rank $p^n$ as an $\sO_{X'}$-module (via $F_{X/S}^\sharp$).
\end{lemma}

\begin{proof}
  The question is local, so we may $f$ assume factors as
  \[
    X \lblto{g} \lA^n_S \to S
  \]
  for an \'{e}tale map $g$. Setting $Y = \lA^n_S$, we have the commutative diagram
  \[
    \begin{tikzcd}
      X \ar[dr, "F_{X/S}"] \ar[d, "F_{X/Y}", swap] \\
      X'_Y \ar[r, "F'_{Y/S}"] \ar[d, "g'_Y"] &
      X'_S \ar[r, "G_{X/S}"] \ar[d, "g'_S"] &
      X \ar[d, "g"] \\
      Y \ar[r, "F_{Y/S}"] \ar[dr, "h"] &
      Y' \ar[r, "G_{Y/S}"] \ar[d, "h'"] &
      Y \ar[d, "h", swap] \\
      &
      S \ar[r, "F_S"] &
      S
    \end{tikzcd}
  \]
  in which all squares are pullbacks. Since $F_{X/Y}$ is an isomorphism by \cref{degen--cartier--frobenius-etale}, this reduces us to the case $X = Y = \lA^n_S$, for which the claim follows from   \cref{degen--cartier--frobenius-eg}.
\end{proof}

We now discuss the behavior of the de Rham complex in characteristic $p$.

\begin{nothing}
  \label{degen--cartier--kahler-base-change}
  As $f' \c X' \to S$ is obtained by a base-change of $f \c X \to S$, the natural map $G_{X/S}^*\Omega_{X/S}^1 \to \Omega_{X'/S}$ is an isomorphism. It follows that $\Omega_{X'/S}$ is given by $\sO_{X'} \otimes_{\sO_X} \Omega_{X/S}^1$. We thus denote by $1 \otimes dt$ the image of a local section $dt$ of $\Omega_{X/S}^1$ in the natural map $\Omega_{X/S}^1 \to \Omega^1_{X'/S}$.
\end{nothing}

\begin{lemma}
  \label{degen--cartier-linear}
  \begin{enumerate}[leftmargin=*]
  \item The image of the map $F_{X/S}^\sharp \c \sO_{X'} \to \sO_X$ is contained in the kernel of the differential $d \c \sO_x \to \Omega_{X/S}^1$.
  \item The differential of the de Rham complex $\Omega_{X/S}^\bullet$ is $\sO_{X'}$-linear. Thus the cochain-complex is a graded-commutative differential-graded $\sO_{X'}$-algebra, and hence its cohomology sheaves $\bigoplus_i \rH^i(\Omega_{X/S}^\bullet)$ form a graded-commutative graded $\sO_{X'}$-algebra.
  \end{enumerate}
\end{lemma}

\begin{proof}
  The first statement follows from the description of $F_{X/S}^\sharp$ in   \cref{degen--cartier--frobenius-id}, as $d$ is by definition $f^{-1}\sO_S$-linear and $d(t^p) = pt^{p-1} = 0$ for $t$ a local section of $\sO_X$. The second statement follows from the first by the Leibniz rule.
\end{proof}

\begin{theorem}[Cartier isomorphism]
  \label{degen--cartier--main}
  \begin{enumerate}[leftmargin=*]
  \item There exists a map of graded $\sO_{X'}$-algebras
    \[
      \gamma = \bigoplus_i \gamma^i \c \bigoplus_i \Omega^i_{X'/S} \to \bigoplus_i \rH^i(\Omega_{X/S}^\bullet)
    \]
    uniquely characterized by its behavior in degree $1$, where $\gamma^1$ sends the section $1 \otimes dt$ of $\Omega^1_{X'/S}$ (notation as in \cref{degen--cartier--kahler-base-change}) to the class $[t^{p-1}dt]$ in $\rH^1(\Omega_{X/S}^\bullet)$.
  \item If $f$ is smooth, $\gamma$ is an isomorphism.
\end{enumerate}
\end{theorem}

\begin{proof}[Proof sketch]
  \begin{enumerate}[leftmargin=*]
  \item That the map $\gamma$ is uniquely determined by $\gamma^1$ is immediate from the fact that the left-hand side is precisely the exterior algebra over $\sO_{X'}$, i.e. the free graded-commutative graded $\sO_{X'}$-algebra, on $\Omega_{X'/S}^1$. To define $\gamma^1$ we use the universal property of $\Omega_{X'/S}^1$: one shows that the map $\sO_{X'} \to \rH^1(\Omega_{X/S}^\bullet)$ sending the section $a \otimes t$ (notation as in \cref{degen--cartier--frobenius-id}) to the class $[at^{p-1}dt]$ is an $(f')^{-1}\sO_S$-linear derivation on $\sO_{X'}$.

  \item The question is local, so we may assume $f$ factors as a composite
    \[
      X \lblto{g} \lA_S^n \to S
    \]
    where $g$ is \'{e}tale. Using similar reasoning as in \cref{degen--cartier--frobenius-smooth}, one can reduce to the case $X = \lA_S^n$. Then using a Kunneth formula one can reduce to the case $n=1$, for which we simply look at our motivating computation from \cref{degen--strategy}. \qedhere
  \end{enumerate}
\end{proof}

%%%%%%%%%%%%%%%%%%%%%%%%%%%%%%%%%%%%%%%%%%%%%%%%%%%%%%%%%%%%%%%%%%%%%%

\subsection{Lifting and degeneration in positive characteristic}
\label{degen--lift}

In this subsection we assume our map $f \c X \to S$ is smooth, and we investigate when the Cartier isomorphism \cref{degen--cartier--main} can be lifted to an equivalence
\[
\bigoplus_i \Omega_{X'/S}^i[-i] \iso \Omega_{X/S}^\bullet
\]
in the derived category $D(X')$. We first make a reduction allowing us to focus on the degree $i = 1$.

\begin{lemma}
  \label{degen--lift--extend}
  Suppose there is a map $\phi^1 \c \Omega_{X'/S}^1[-1] \to \Omega_{X/S}^\bullet$ in $D(X')$ which induces the Cartier isomorphism $\gamma^1$ in degree $1$ cohomology. Then there is a map $\phi^i \c \Omega_{X'/S}^i[-i] \to \Omega_{X/S}^\bullet$ inducing the Cartier isomorphism $\gamma^i$ in degree $i$ cohomology for all $0 \le i < p$.
\end{lemma}

\begin{proof}
  Firstly, we always have $\phi^0 \ce F_{X/S}^\sharp \c \sO_{X'} \to \sO_X$ inducing the Cartier isomorphism in degree $0$ by definition, so we just need to worry about $2 \le i < p$.

  Since $X'$ and $X$ are smooth over $S$, $\Omega_{X'/S}^i$ is a locally free $\sO_{X'}$-module and $\Omega_{X/S}^i$ is a locally free $\sO_X$-module; but by \cref{degen--cartier--frobenius-smooth}, $\sO_X$ is a locally free $\sO_{X'}$-module, so $\Omega_{X/S}^i$ is locally free as an $\sO_{X'}$-module as well. It follows that the $i$-fold \emph{derived} tensor powers in $D(X')$ of $\Omega_{X'/S}^1[-1]$ and $\Omega_{X/S}^\bullet$ are their ordinary tensor powers. Thus taking the $i$-fold derived tensor power of $\phi^1$ gives us a map
  \[
    (\phi^1)^{\otimes i} \c \Omega_{X'/S}^{\otimes i}[-i] \to (\Omega_{X/S}^\bullet)^{\otimes i}
  \]
  in $D(X')$. We can compose with the product map $(\Omega_{X/S}^\bullet)^{\otimes i} \to \Omega_{X/S}^\bullet$ on the right-hand side, and on the left-hand side we can precompose with the standard map
  \[
    \Omega_{X'/S}^i \to \Omega_{X'/S}^{\otimes i}, \quad
    \omega_1 \wedge \cdots \wedge \omega_i \mapsto \frac{1}{i!} \sum_{\sigma \in \rA_i} \operatorname{sgn}(\sigma)\, \omega_{\sigma(1)} \otimes \cdots \otimes \omega_{\sigma(i)},
  \]
  which makes sense for $i < p$ in characteristic $p$, begetting our desired map
  \[
    \phi^i \c \Omega_{X'/S}^i[-i] \to \Omega_{X/S}^\bullet.
  \]
  That $\phi^i$ in fact induces the Cartier isomorphism $\gamma^i$ follows from the fact that $\gamma^i$ was defined by extending $\gamma^1$ multiplicatively.
\end{proof}

Now, the idea for constructing the desired $\phi^1 \c \Omega_{X'/S}^1[-1] \to \Omega_{X/S}^\bullet$ is to try to interpret the cohomology class $[t^{p-1}dt]$ defining the Cartier isomorphism $\gamma^1$ as the class of a cycle $\frac{1}{p} d(t^p)$. This of course does not make sense when we're working modulo $p$, but it can make sense if we can lift ourselves to a situation modulo $p^2$.

\begin{proposition}
  \label{degen--lift--main}
  Suppose given a commutative diagram
  \[
    \begin{tikzcd}
      X' \ar[r] \ar[d, "f'", swap] &
      Y' \ar[d, "g'"] \\
      S \ar[r] \ar[d, "u", swap] &
      T \ar[d, "v"] \\
      \Spec \lF_p \ar[r] &
      \Spec \lZ/p^2.
    \end{tikzcd}
  \]
  in which each square is a pullback, $g$ is smooth, and $v$ is flat. Then there exists a map $\phi^1 \c \Omega_{X'/S}^1[-1] \to \Omega_{X/S}^\bullet$ in $D(X')$ inducing the Cartier isomorphism $\gamma^1$ in degree $1$ cohomology.
\end{proposition}

\begin{proof}[Proof sketch]
  First one assumes one has the additional data $D$ of a diagram
  \[
    \begin{tikzcd}
      X \ar[r] \ar[d, "f", swap] &
      Y \ar[d, "g"] \\
      S \ar[r] \ar[d, "u", swap] &
      T \ar[d, "v"] \\
      \Spec \lF_p \ar[r] &
      \Spec \lZ/p^2.
    \end{tikzcd}
  \]
  in which each square is a pullback, as well as a map of $T$-schemes $F \c Y \to Y'$ such that the square
  \[
    \begin{tikzcd}
      X \ar[r] \ar[d, "F_{X/S}", swap] &
      Y \ar[d, "F"] \\
      X' \ar[r]  &
      Y'
    \end{tikzcd}
  \]
  commutes. Given such data, we may construct a commutative diagram
  \[
    \begin{tikzcd}
      \Omega^1_{Y'/T} \ar[r, "F^\sharp"] \ar[d] &
      \Omega^1_{Y/T}  \\
      \Omega^1_{X'/S} \ar[r]  &
      \Omega^1_{X/S} \ar[u, "{[p]}", swap]
    \end{tikzcd}
  \]
  where the right-hand vertical map is multiplication by $p$. The bottom map, which should be thought of as obtained by dividing $F^\sharp$ by $p$, gives us our required map $\phi^1_D$.

  One then shows that, provided two different sets of this data $D$ and $D'$, the resulting maps $\phi^1_D$ and $\phi^1_{D'}$ differ by a canonical homotopy. Finally, one uses smoothness to obtain the data $D$ locally on $X$, and glues to obtain the global map $\phi^1$. (This gluing can be done formally if we work with the derived category $D(X')$ as an $\infty$-category.)
\end{proof}

\begin{nothing}
  Suppose $S = \Spec \kappa$ for $\kappa$ a perfect field of characteristic $p$. Note that $\kappa$ being perfect means that $F_S$ is an isomorphism, and hence its base-change $G_{X/S}$ is an isomorphism of $S$-schemes $X' \iso X$.
  \begin{enumerate}
  \item Then there is a unique flat lift $T = \Spec W_2(\kappa)$ of $S$ over $\lZ/p^2$, where $W_2(k)$ denotes the \emph{ring of Witt vectors of length $2$} over $\kappa$.
  \item We say the $\kappa$-scheme $X$ can be \emph{lifted over $W_2(\kappa)$} if a smooth $T$-scheme $Y'$ exists as in the hypothesis of \cref{degen--lift--main}.
  \end{enumerate}
\end{nothing}

\begin{theorem}
  \label{degen--lift--degen}
  Let $\kappa$ a perfect field of characteristic $p$. Let $X$ be smooth, proper $\kappa$-scheme of dimension $< p$. If $X$ can be lifted over $W_2(\kappa)$, then the Hodge--de Rham spectral sequence
  \[
    E_1^{i,j} = \rH^j(X;\Omega_{X/\kappa}^i) \quad \Rightarrow \quad
    \rH_\dR^{i+j}(X/\kappa)
  \]
  degenerates at $E_1$.
\end{theorem}

\begin{proof}
  We implement our strategy \cref{degen--strategy}. It suffices to show that
  \[
    \dim_\kappa \rH_\dR^k(X/\kappa) = \sum_{i+j=k} \dim_\kappa \rH^j(X;\Omega_{X/\kappa}^i),
  \]
  where all dimensions are finite since $X$ is proper. Let $S \ce \Spec \kappa$; since $\kappa$ is perfect, the Frobenius $F_S$ is an isomorphism, and hence we have a base change isomorphism
  \[
    F_S^*\rH^j(X';\Omega_{X'/\kappa}^i) \iso \rH^j(X;\Omega_{X/\kappa}^i) \implies \rH^j(X';\Omega_{X'/\kappa}^i) = \dim_\kappa \rH^j(X;\Omega_{X/\kappa}^i).
  \]
  By \cref{degen--lift--main} and \cref{degen--lift--extend}, the hypotheses that $X$ can be lifted over $W_2(\kappa)$ and has dimension $< p$ implies that we have an equivalence
  \[
    \bigoplus_i \Omega_{X'/k}^k[-i] \iso \Omega_{X/k}^\bullet
  \]
  in $D(X')$; after applying sheaf cohomology we obtain an isomorphism
  \[
    \bigoplus_{i+j=k} \rH^j(X;\Omega_{X/\kappa}^i) = \rH_\dR^k(X/\kappa)
  \]
  Putting everything we've said together proves the claim.
\end{proof}

%%%%%%%%%%%%%%%%%%%%%%%%%%%%%%%%%%%%%%%%%%%%%%%%%%%%%%%%%%%%%%%%%%%%%%

\subsection{Degeneration in characteristic zero}
\label{degen--final}

Finally, we sketch how our positive characteristic degeneration result implies the desired characteristic zero result, using standard limit techniques.

\begin{proof}[Proof of \cref{degen--main}]
  Let $h^{i,j} \ce \dim_K \rH^j(X;\Omega_{X/K}^i)$ and $h^k \ce \dim_K \rH_\dR^k(X/K)$. As in \cref{degen--lift--degen}, properness of $X$ implies these are all finite, and it suffices to show $h^k = \sum_{i+j=k} h^{i,j}$.

  Writing $K$ as a filtered colimit of its finitely generated $\lZ$-subalgebras, we may find a pullback diagram
  \[
    \begin{tikzcd}
      X \ar[r] \ar[d, "f", swap] &
      Y \ar[d, "g"] \\
      \Spec K \ar[r] &
      \Spec A
    \end{tikzcd}
  \]
  where $A$ is a finitely generated $\lZ$-algebra and $g$ is still smooth and proper. Passing to a (dense) open subset of $\Spec A$ if necessary, we may assume that $\Spec A$ is smooth over $\Spec \lZ$ and that the sheaves $\rR^jg_*\Omega_{Y/A}^i$ and $\rR^kg_*\Omega_{Y/A}^\bullet$ are free of constant rank; by cohomology and base change, these ranks must in fact be $h^{i,j}$ and $h^k$, respectively.

  Let $\Spec \kappa \to \Spec A$ be a closed point. Since $A$ is finitely generated over $\lZ$, $\kappa$ is necessarily finite field. Moreover, by passing to the dense affine open $\Spec A[1/N]$ for any $N \in \lN$, we may assume $\kappa$ has characteristic as large as we desire. In particular, since $Y$ is proper and hence quasi-compact, we may assure that $\kappa$ has characteristic $p$ larger than the dimension of all fibers of $g$. Since $\Spec A$ is smooth over $\Spec \lZ$, the map $\Spec \kappa$ can be extended to the nilpotent thickening $\Spec W_2(\kappa)$. We may thus form the following pullback squares
  \[
    \begin{tikzcd}
      Z \ar[r] \ar[d, "h"] &
      Z' \ar[r] \ar[d, "h'"] &
      Y \ar[d, "g"] \\
      \Spec \kappa \ar[r] &
      \Spec W_2(\kappa) \ar[r] &
      \Spec A.
    \end{tikzcd}
  \]
  Again by cohomology and base change, we have
  \[
    \dim_\kappa \rH^j(Z;\Omega^i_{Z/\kappa'}) = h^{i,j}
    \quad\text{and}\quad
    \dim_\kappa \rH_\dR^k(Z/\kappa) = h^k.
  \]
  But since $Z$ is lifted over $W_2(\kappa)$ and has dimension $< p$ by construction, \cref{degen--lift--degen} implies $h^k = \sum_{i+j=k} h^{i,j}$, as desired.
\end{proof}