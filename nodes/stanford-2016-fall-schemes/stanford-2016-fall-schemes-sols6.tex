\input{preamble}

\title{Math 216A Homework 6}
\author{Arpon Raksit}
\date{November 3, 2016}

\numberwithin{block}{section}

%%%%%%%%%%%%%%%%%%%%%%%%%%%%%%%%%%%%%%%%%%%%%%%%%%%%%%%%%%%%%%%%%%%%%%

\begin{document}
\maketitle

\newcommand{\codim}{\operatorname{codim}}

%%%%%%%%%%%%%%%%%%%%%%%%%%%%%%%%%%%%%%%%%%%%%%%%%%%%%%%%%%%%%%%%%%%%%%

\section{Dimension pathologies}

Let $R$ be a discrete valuation ring containing its residue field $k$.

Let $X \ce \Spec R[t]$. Let $\pi$ be a uniformizer in $R$ and $\kp \ce (\pi t - 1) \subseteq R[t]$. Observe that $R[t]/\kp = R[t]/(\pi t - 1) \iso R_\pi$ is the fraction field of $R$, in particular a field, so $\kp$ is maximal. Thus $Y \ce \{\kp\} \subseteq X$ is a closed set of dimension $0$. Then observe that $Y$ has codimension $1$, i.e. $\dim R[t]_\kp = 1$, by Krull's hauptidealsatz. Since $\dim R = 1 \implies \dim R[t] = 2$, we have that
\[
\dim R[t]_\kp = 1 < 2 = \dim X \quad\text{and}\quad
\dim Y + \codim(Y,X) = 1 < 2 = \dim X
\]
(note all we really needed here is that $\dim R[t] \ge 2$, which is witnessed by the chain of primes $0 \subsetneq (\pi) \subsetneq (\pi, t)$).

Lastly, consider the (nonempty) open set $U \ce D(\pi) \subseteq X$, which is isomorphic to $\Spec R[t]_\pi \iso \Spec R_\pi[t]$. Since $R_\pi$ is a field, this has dimension $1$, unlike $X$ which has dimension $2$.

%%%%%%%%%%%%%%%%%%%%%%%%%%%%%%%%%%%%%%%%%%%%%%%%%%%%%%%%%%%%%%%%%%%%%%

\end{document}
