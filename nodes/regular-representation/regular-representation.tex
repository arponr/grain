\input{preamble}

\title{The regular representation}
\author{Arpon Raksit}
\date{December 30, 2015}

\numberwithin{equation}{section}

\begin{document}
\maketitle

\newcommand{\chr}{\operatorname{char}}

%%%%%%%%%%%%%%%%%%%%%%%%%%%%%%%%%%%%%%%%%%%%%%%%%%%%%%%%%%%%%%%%%%%%%%

\section{Finite groups}

Let $G$ be a finite group. Let $k$ be an algebraically closed field such that $\chr(k)$ does not divide $G$. All vector spaces and representations in this section are over $k$.

\begin{definition}
  \label{finite-regular-rep}
  Consider the vector space $\Fun(G)$ of functions from $\phi \c G \to k$. This has the structure of a $G \times G$-representation by letting $(g,h) \in G \times G$ act on $\phi \c G \to k$ via the formula
  \[
    ((g,h)\cdot \phi)(x) = \phi(g^{-1}xh).
  \]
  This is perhaps what should be called the \emph{regular representation} associated to $G$. Well, note that it's a representation of $G \times G$, but it then naturally induces $G$-representation structures on $\Fun(G)$:
  \begin{enumerate}
  \item \label{finite-left-regular-rep} restricting in the map $(\id_G, 1) \c G \to G \times G$ sending $g \mapsto (g,1)$ gives us the \emph{left regular representation}, where $(g \cdot \phi)(x) = \phi(g^{-1}x)$;
  \item \label{finite-right-regular-rep} restricting in the map $(1, \id_G) \c G \to G \times G$ sending $g \mapsto (1,g)$ gives us the \emph{right regular representation}, where $(g \cdot \phi)(x) = \phi(xg)$.
  \end{enumerate}
  Unless otherwise stated we consider $\Fun(G)$ as a $G$-representation using the left regular representation structure.
\end{definition}

\begin{lemma}
  \label{maps-to-regular}
  Let $V$ be a representation of $G$. Then there is a natural isomorphism (of vector spaces)
  \[
    \Hom_G(V, \Fun(G)) \isoto V^\vee,
  \]
  given by sending $T \c V \to \Fun(G)$ to the linear functional $v \mapsto T(v)(1)$.
\end{lemma}

\begin{proof}
  Trivial.
\end{proof}

\begin{corollary}
  \label{regular-decomposition}
  Let $\{V_i\}_{i \in I}$ be a set of representatives for the isomorphism classes of irreducible representations of $G$. Then there is an isomorphism of $G$-representations
  \[
    \Fun(G) \iso \bigoplus_{i \in I} V_i^\vee \otimes V_i
  \]
  (where here $V_i^\vee$ is viewed just as a vector space, i.e. a trivial $G$-representation). In particular, taking dimensions of each side, we get
  \[
    |G| = \sum_{i \in I} \dim(V_i)^2.
  \]
\end{corollary}

\begin{proof}
  For any representation $V$ we have by semisimplicity and Schur's lemma that
  \[
    V \iso \bigoplus_{i \in I} \Hom_G(V_i, V) \otimes V_i.
  \]
  The claim then follows from plugging in $V = \Fun(G)$ and applying \cref{maps-to-regular}.
\end{proof}

%%%%%%%%%%%%%%%%%%%%%%%%%%%%%%%%%%%%%%%%%%%%%%%%%%%%%%%%%%%%%%%%%%%%%% 

\section{Compact Lie groups}

...

%%%%%%%%%%%%%%%%%%%%%%%%%%%%%%%%%%%%%%%%%%%%%%%%%%%%%%%%%%%%%%%%%%%%%%

% \bibliographystyle{amsalpha}
% \bibliography{refs}

\end{document}
