\input{preamble}

\title{Measure theory}
\author{Arpon Raksit}
\date{January 18, 2016}

\numberwithin{equation}{section}

\begin{document}
\maketitle

%%%%%%%%%%%%%%%%%%%%%%%%%%%%%%%%%%%%%%%%%%%%%%%%%%%%%%%%%%%%%%%%%%%%%%

\section{Measure spaces}

\begin{notation}
  We denote the power set of a set $\Omega$ by $\cP(\Omega)$.
\end{notation}

\begin{definitions}
  \begin{enumerate}
  \item \label{sigma-algebra}
    Let $\Omega$ be a set. A \emph{$\sigma$-algebra} on $\Omega$ is a collection $\cF \subseteq \cP(\Omega)$ of subsets of $\Omega$ satisfying:
    \begin{enumerate}
    \item \label{empty-measurable} $\emptyset \in \cF$.
    \item \label{complement-measurable} If $A \in \cF$, then $\Omega - A \in \cF$.
    \item \label{union-measurable}
      If $\{A_n\}_{n \in \lN} \subseteq \cF$, then $\bigcup_{n \in \lN} A_n \in \cF$.
    \end{enumerate}
    Note that applying \cref{complement-measurable} to \cref{empty-measurable} and \cref{union-measurable} implies, respectively:
    \begin{enumerate}[resume]
    \item \label{whole-measurable} $\Omega \in \cF$.
    \item \label{intersection-measurable}
      If $\{A_n\}_{n \in \lN} \subseteq \cF$, then $\bigcap_{n \in \lN} A_n \in \cF$.
    \end{enumerate}

    If $\Omega$ is equipped with a $\sigma$-algebra $\cF$, then we say a subset $A \subseteq \Omega$ is \emph{measurable} if $A \in \cF$.

  \item \label{measure}
    Let $\cF$ be a $\sigma$-algebra on a set $\Omega$. Then a \emph{measure} on $\cF$ (when the $\sigma$-algebra $\cF$ is understood/implicit, we will also absusively call this a measure on $\Omega$) is a function $\mu \c \cF \to [0, \infty]$ that is countably additive, i.e.:
    \begin{enumerate}
    \item $\mu(\emptyset) = 0$.
    \item If $\{A_n\}_{n \in \lN} \subseteq \cF$ is a collection of (pairwise) disjoint measurable sets, then $\mu\l(\bigcup_{n \in \cN} A_n\r) = \sum_{n \in \lN} \mu(A_n)$.
    \end{enumerate}
    Here addition with $\infty$ is treated as one would expect.

  \item \label{measure-space}
    A \emph{measure space} is a triple $(\Omega, \cF, \mu)$ where $\Omega$ is a set, $\cF$ a $\sigma$-algebra on $\Omega$, and $\mu$ a measure on $\cF$.
  \end{enumerate}
\end{definitions}

\begin{lemma}
  \label{measure-monotone}
  Let $(\Omega, \cF, \mu)$ be a measure space. Suppose $A,B \in \cF$ such that $A \subseteq B$. Then $\mu(A) \le \mu(B)$.
\end{lemma}

\begin{proof}
  Note that $B-A = B \cap (\Omega-A) \in \cF$ by \cref{complement-measurable,intersection-measurable}. Then $\mu(B) = \mu(A) + \mu(B-A)$ by countable additivity. This proves the claim since, by definition, $\mu(B-A) \ge 0$.
\end{proof}

\begin{definition}
  \label{finite-measure}
  We say a measure space $(\Omega,\cF,\mu)$ is \emph{finite} if $\mu(\Omega) < \infty$. By \cref{measure-monotone}, this condition implies $\mu(A) < \infty$ for all $A \in \cF$.
\end{definition}

\begin{lemma}
  \label{intersect-sigmas}
  Let $\Omega$ be a set and let $\{\cF_i\}$ be a collection of $\sigma$-algebras on $\Omega$. Then $\bigcap \cF_i$ is also a $\sigma$-algebra on $\Omega$.
\end{lemma}

\begin{proof}
  Evident.
\end{proof}

\begin{lemma}
  \label{smallest-sigma}
  Let $\Omega$ be a set and let $\cA \subseteq \cP(\Omega)$ be any collection of subsets of $\Omega$. Then there is a minimal (with respect to inclusion) $\sigma$-algebra $\cF$ on $\Omega$ containing $\cA$. We refer to $\cF$ as the $\sigma$-algebra \emph{generated by} $\cA$.
\end{lemma}

\begin{proof}
  Let $\{\cF_i\}$ be the collection of $\sigma$-algebras on $\Omega$ containing $\cA$; the collection is certainly nonempty, as it contains the $\sigma$-algebra $\cP(\Omega)$ of \emph{all} subsets of $\Omega$. Then the desired minimal $\sigma$-algebra is easily seen to be the intersection  $\bigcap \cF_i$, which is a $\sigma$-algebra by \cref{intersect-sigmas}.
\end{proof}

\begin{definition}
  Let $X$ be a topological space. The \emph{Borel $\sigma$-algebra} on $X$, often denoted $\cB$, is the $\sigma$-algebra generated by the collection of open sets in $X$, often denoted $\cG$.
\end{definition}

%%%%%%%%%%%%%%%%%%%%%%%%%%%%%%%%%%%%%%%%%%%%%%%%%%%%%%%%%%%%%%%%%%%%%%

% \bibliographystyle{amsalpha}
% \bibliography{refs}

\end{document}
